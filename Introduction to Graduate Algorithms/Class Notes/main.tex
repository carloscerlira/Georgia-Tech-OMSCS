\documentclass{article}
\usepackage[utf8]{inputenc}
\textheight = 25cm 
\textwidth = 16cm
\topmargin = -3.0cm 
\oddsidemargin = 1cm
\usepackage{hyperref}
\hypersetup{
    colorlinks=true,
    linkcolor=blue,
    filecolor=blue,
    citecolor=black,      
    urlcolor=blue,
    }

\usepackage{float}
\usepackage{graphicx}

\usepackage{gensymb}

\usepackage{amsmath}
\DeclareMathOperator*{\argmax}{argmax} % thin space, limits 
\usepackage{amssymb}
\usepackage{amsfonts}
\usepackage{mathtools, xparse}
\usepackage[shortlabels]{enumitem}

\usepackage[many]{tcolorbox}
\usepackage{lipsum}
\usepackage{amssymb}

\title{Introduction to Graduate Algorithms}
\author{Cerritos Lira, Carlos}
\date{23 de junio del 2020}

\newcommand{\pr}[1]{\left(#1\right)}
\newcommand{\pt}[2]{\dfrac{\partial #1}{\partial #2}}

\begin{document}
\maketitle

\section*{FFT}
Problem: given $A$ and $B$ two polynomias of degree $n-1$ find an 
algorithm to compute $C = A \cdot B$ in time $O(n \ln n)$. 
\subsubsection*{Computing polynomial convolution}
Given $A$ and $B$, find $x_1,...x_{2n}$ points and define:
\begin{enumerate}
    \item Define $A_{even}(y)$ and $A_{odd}(y)$ where $deg \leq \tfrac{n}{2}-1$.
    \item Recursively evaluate at $n$ points $y_{i} = x_i^2$.
    \item In order $O(n)$ time get $A(x_i)$.
\end{enumerate}

\end{document}